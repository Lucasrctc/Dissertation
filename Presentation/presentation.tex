\documentclass[professionalfonts]{beamer}
%
% Choose how your presentation looks.
%
% For more themes, color themes and font themes, see:
% http://deic.uab.es/~iblanes/beamer_gallery/index_by_theme.html
%
\mode<presentation>
{
  \usetheme{Luebeck}      % or try Darmstadt, Madrid, Warsaw, ...
  \usecolortheme{whale} % or try albatross, beaver, crane, ...
  \usefonttheme{serif}  % or try serif, structurebold, ...
  \setbeamertemplate{navigation symbols}{}
  \setbeamertemplate{caption}[numbered]
  \setbeamertemplate{page number in head/foot}[totalframenumber]
  \setbeamertemplate{headline}{} % Removes the headlines
  \setbeamercolor{footline}{fg=blue}
  %uncomment if needed to raise slide number higher
%\setbeamerfont{footline}{series=\bfseries}
%  \addtobeamertemplate{navigation symbols}{}{%
%    \usebeamerfont{footline}%
%    \usebeamercolor[fg]{footline}%
%    \hspace{1em}%
%    \insertframenumber/\inserttotalframenumber
%}
} 

%%%%%%%%%%%%%%%%%%%%%%%%%%%%%%%%%%%%%%%%%%%%%%
%Include any other add-on  packages you need:%
%%%%%%%%%%%%%%%%%%%%%%%%%%%%%%%%%%%%%%%%%%%%%%
\usepackage{amsmath,graphicx,mathtools, amssymb}
\allowdisplaybreaks
\usepackage{booktabs} % Required for better horizontal rules in tables
\usepackage{listings} % Required for insertion of code
\usepackage[utf8]{inputenc}
\usepackage[T1]{fontenc}
\usepackage{lmodern}
\usepackage[labelfont=bf,labelsep=period]{caption}
\usepackage{multirow}
\usepackage{textcomp}
\usepackage{hyperref}
\usepackage{afterpage}
\usepackage{pdflscape}
\usepackage{hhline}
%\usepackage{enumitem}
\usepackage{bm}
\usepackage{datetime}
\usepackage{xstring} % conditionals
\usepackage{xcolor}
\usepackage{movie15}
\usepackage{bm}
\usepackage{natbib}

\title[Dissertation defence]{New quantum computing techniques applied to chemistry dynamics}
\author{Lucas Ramos Cardoso Tin\^{o}co Cortez}
\institute{Texas Tech University}
\date{\today}

%%%%%%%%%%%%%%%%%%%%%%%%%%%%%%%%%%%%%%%%%%%%%%%
%Uncomment if the grad school doesn't like the%
%line under the  running head:                %
%%%%%%%%%%%%%%%%%%%%%%%%%%%%%%%%%%%%%%%%%%%%%%%
%\renewcommand{\headrulewidth}{0pt}


%%%%%%%%%%%%%%%%%%%%%%%%%%%%%%%%%%%%%%%%%%%%%%%%%%
%Spacing -- Do you want double or one-and-a-half?%
%%%%%%%%%%%%%%%%%%%%%%%%%%%%%%%%%%%%%%%%%%%%%%%%%%
%\doublespacing
%\onehalfspacing
%%%%%%%%%%%%%%%%%%%%%%%%%%%%%%%%%%%%%%%%%%%%%%%%%%%%%%%%%%%%%
%Leave the one you want uncommented.                        %
%In places where single-line-spacing is appropriate         %
%e.g, extended quotations, you can enclose the material     %
%in a singlespacing environment (with \begin{singlespacing} %
% ...  \end{singlespacing}                                  %
%%%%%%%%%%%%%%%%%%%%%%%%%%%%%%%%%%%%%%%%%%%%%%%%%%%%%%%%%%%%%


%%%%%%%%%%%%%%%%%%%%%%%%%%%%%%%%%%%%%%%%%%%%%%%%%%
%Other preamble stuff, e.g., theorem environments%
%or newcommands go here:                         %
% e.g.                                           %
%%%%%%%%%%%%%%%%%%%%%%%%%%%%%%%%%%%%%%%%%%%%%%%%%%
% \newtheorem{theorem}{Theorem}
% \newtheorem{proposition}[theorem]{proposition}
% \newtheorem{question}{Question}
% \newtheorem{conjecture}{Conjecture}
%\newcommand\Set[2]{\left\{\,#1\middle\mid#2\,\right\}}
\newcommand{\bx}{\bm{x}}
\newcommand{\br}{\bm{r}}
\newcommand{\bra}[1]{\ensuremath{\left\langle#1\right\vert}}
\newcommand{\ket}[1]{\ensuremath{\left|#1\right\rangle}}
\newcommand{\braket}[2]{\left< #1 \middle\vert #2 \right>}
\newcommand{\sandwich}[3]{\left< #1 \middle\vert #2 \middle\vert #3 \right>}
%\newcommand{\norm}[1]{\left\lVert #1 \right\rVert}
%\newcommand{\abs}[1]{\lVert #1 \rVert}
\newcommand{\inpr}[1]{\left< #1 \right>}
\newcommand{\s}[1]{\sin\left( #1 \right)}
\newcommand{\ssq}[1]{\sin^2\left( #1 \right)}
\newcommand{\co}[1]{\cos\left( #1 \right)}
\newcommand{\cosq}[1]{\cos^2\left( #1 \right)}
\newcommand{\av}[1]{\left< #1 \right>}
\newcommand{\her}[2]{H_{#1}\left( #2 \right)}
\newcommand{\herv}[1]{H_{\nu}\left( #1 \right)}
%\renewcommand{\labelenumi}{\Roman{enumi} -}
\newcommand{\ex}[1]{\exp\left( #1 \right)}
\newcommand{\closepar}{)}
\newcommand{\paren}[1]{\left( #1 \right)}
\newcommand{\ddx}{\frac{d}{dx}}
\newcommand{\ddt}{\frac{d}{dt}}
\newcommand{\pdx}{\frac{\partial}{\partial x}}
\newcommand{\pdt}{\frac{\partial}{\partial t}}
\newcommand{\dd}[1]{\frac{d}{d#1}}
\newcommand{\pd}[1]{\frac{\partial}{\partial #1}}
\newcommand{\fpd}[2]{\frac{\partial #1}{\partial #2}}
\newcommand{\pp}[1]{\frac{\partial \Psi}{\partial #1}}
\newcommand{\kpp}[1]{\frac{\partial \ket\Psi}{\partial #1}}
\newcommand{\bpp}[1]{\frac{\partial \bra\Psi}{\partial #1}}
\newcommand{\kppn}[2]{\frac{\partial^{#2} \ket\Psi}{\partial {#1}^{#2}}}
\newcommand{\bppn}[2]{\frac{\partial^{#2} \bra\Psi}{\partial {#1}^{#2}}}
\newcommand{\kppd}[2]{\frac{\partial^{2} \ket\Psi}{\partial #1 \partial #2}}
\newcommand{\bppd}[2]{\frac{\partial^{2} \bra\Psi}{\partial #1 \partial #2}}
\newcommand{\ddn}[2]{\frac{d^{#1}}{d#2^{#1}}}
\newcommand{\pdn}[2]{\frac{\partial^{#1}}{\partial#2^{#1}}}
%\newcommand{\orb}{N}
%\newcommand{\elec}{n}
\newcommand{\elec}{N}
\newcommand{\nuc}{N_{N}}
\newcommand{\orb}{K}
\newcommand{\Ld}[1]{\hat L_{#1}}
\newcommand{\anio}[1]{a_{\oo{#1}}}
\newcommand{\aniu}[1]{a_{\uo{#1}}}
\newcommand{\anig}[1]{a_{\go{#1}}}
\newcommand{\creo}[1]{a^\dagger_{\oo{#1}}}
\newcommand{\creu}[1]{a^\dagger_{\uo{#1}}}
\newcommand{\creg}[1]{a^\dagger_{\go{#1}}}
\newcommand{\vac}{\ket{vac}}
\newcommand{\anticom}[2]{\left\{ #1, #2 \right\}}
\newcommand{\ind}[1]{{\uo #1 \oo #1}}
\newcommand{\Paulid}[1]{
		\IfEqCase{#1}{
			{ii}{I \otimes I}
			{ix}{I \otimes X}
			{iy}{I \otimes Y}
			{iz}{I \otimes Z}
			{xi}{X \otimes I}
			{xx}{X \otimes X}
			{xy}{X \otimes Y}
			{xz}{X \otimes Z}
			{yi}{Y \otimes I}
			{yx}{Y \otimes X}
			{yy}{Y \otimes Y}
			{yz}{Y \otimes Z}
			{zi}{Z \otimes I}
			{zx}{Z \otimes X}
			{zy}{Z \otimes Y}
			{zz}{Z \otimes Z}
		}[\PackageError{p2d}{Undefined option to uo: #1}{}]
}
\newcommand{\uo}[1]{
		\IfEqCase{#1}{
			{0}{\mu}
			{1}{\nu}
			{2}{\xi}
		}[\PackageError{uo}{Undefined option to uo: #1}{}]
}
\newcommand{\oo}[1]{
		\IfEqCase{#1}{
			{0}{\alpha}
			{1}{\beta}
			{2}{\gamma}
		}[\PackageError{unocorb}{Undefined option to unuo: #1}{}]
}
\newcommand{\go}[1]{
		\IfEqCase{#1}{
			{0}{\zeta}
			{1}{\eta}
			{2}{\iota}
			{3}{\kappa}
		}[\PackageError{unocorb}{Undefined option to go: #1}{}]
}
\newcommand{\psiu}[1]{
	\psi_{\uo{#1}}
}
\newcommand{\psio}[1]{
	\psi_{\oo{#1}}
}
\newcommand{\psig}[1]{
	\psi_{\go{#1}}
}

\newcommand{\X}{\begin{bmatrix}	0 & 1 \\ 1 & 0 \end{bmatrix} }
\newcommand{\Y}{\begin{bmatrix}	0 & -i \\ i & 0 \end{bmatrix} }
\newcommand{\Z}{\begin{bmatrix}	1 & 0 \\ 0 & -1 \end{bmatrix} }
\newcommand{\I}{\begin{bmatrix}	1 & 0 \\ 0 & 1 \end{bmatrix} }
\newcommand{\Phase}[1]{\begin{bmatrix} 1 & 0 \\ 0 & e^{i#1} \end{bmatrix} }
\newcommand{\nPhase}[1]{\begin{bmatrix} 1 & 0 \\ 0 & e^{-i#1} \end{bmatrix} }
\newcommand{\II}{\begin{bmatrix}
1 & 0 & 0 & 0 \\
0 & 1 & 0 & 0 \\
 0 & 0 & 1 & 0 \\
0 & 0 & 0 & 1
\end{bmatrix}}
\newcommand{\IX}{ \begin{bmatrix}
0 & 1 & 0 & 0 \\
1 & 0 & 0 & 0 \\
 0 & 0 & 0 & 1 \\
0 & 0 & 1 & 0
\end{bmatrix}}
\newcommand{\IY}{ \begin{bmatrix}
0 & -i & 0 & 0 \\
i & 0 & 0 & 0 \\
 0 & 0 & 0 & -i \\
0 & 0 & i & 0
\end{bmatrix}}
\newcommand{\IZ}{ \begin{bmatrix}
1 & 0 & 0 & 0 \\
0 & -1 & 0 & 0 \\
 0 & 0 & 1 & 0 \\
0 & 0 & 0 & -1
\end{bmatrix}}
\newcommand{\XI}{ \begin{bmatrix}
0 & 0 & 1 & 0 \\
0 & 0 & 0 & 1 \\
 1 & 0 & 0 & 0 \\
0 & 1 & 0 & 0
\end{bmatrix}}
\newcommand{\XX}{ \begin{bmatrix}
0 & 0 & 0 & 1 \\
0 & 0 & 1 & 0 \\
 0 & 1 & 0 & 0 \\
1 & 0 & 0 & 0
\end{bmatrix}}
\newcommand{\XY}{ \begin{bmatrix}
0 & 0 & 0 & -i \\
0 & 0 & i & 0 \\
 0 & -i & 0 & 0 \\
i & 0 & 0 & 0
\end{bmatrix}}
\newcommand{\XZ}{ \begin{bmatrix}
0 & 0 & 1 & 0 \\
0 & 0 & 0 & -1 \\
 1 & 0 & 0 & 0 \\
0 & -1 & 0 & 0
\end{bmatrix}}
\newcommand{\YI}{ \begin{bmatrix}
0 & 0 & -i & 0 \\
0 & 0 & 0 & -i \\
 i & 0 & 0 & 0 \\
0 & i & 0 & 0
\end{bmatrix}}
\newcommand{\YX}{ \begin{bmatrix}
0 & 0 & 0 & -i \\
0 & 0 & -i & 0 \\
 0 & i & 0 & 0 \\
i & 0 & 0 & 0
\end{bmatrix}}
\newcommand{\YY}{ \begin{bmatrix}
0 & 0 & 0 & -1 \\
0 & 0 & 1 & 0 \\
 0 & 1 & 0 & 0 \\
-1 & 0 & 0 & 0
\end{bmatrix}}
\newcommand{\YZ}{ \begin{bmatrix}
0 & 0 & -i & 0 \\
0 & 0 & 0 & i \\
 i & 0 & 0 & 0 \\
0 & -i & 0 & 0
\end{bmatrix}}
\newcommand{\ZI}{ \begin{bmatrix}
1 & 0 & 0 & 0 \\
0 & 1 & 0 & 0 \\
 0 & 0 & -1 & 0 \\
0 & 0 & 0 & -1
\end{bmatrix}}
\newcommand{\ZX}{ \begin{bmatrix}
0 & 1 & 0 & 0 \\
1 & 0 & 0 & 0 \\
 0 & 0 & 0 & -1 \\
0 & 0 & -1 & 0
\end{bmatrix}}
\newcommand{\ZY}{ \begin{bmatrix}
0 & -i & 0 & 0 \\
i & 0 & 0 & 0 \\
 0 & 0 & 0 & i \\
0 & 0 & -i & 0
\end{bmatrix}}
\newcommand{\ZZ}{ \begin{bmatrix}
1 & 0 & 0 & 0 \\
0 & -1 & 0 & 0 \\
 0 & 0 & -1 & 0 \\
0 & 0 & 0 & 1
\end{bmatrix}}
\DeclarePairedDelimiter\norm{\lVert}{\rVert}
\DeclarePairedDelimiter\abs{\lvert}{\rvert}

%Presentation macros:
\newcommand{\SubItem}[1]{
    {\setlength\itemindent{15pt} \item[*] #1}
}

%Transition for every section
\AtBeginSection[]
{
  \begin{frame}
    \tableofcontents[currentsection]
  \end{frame}

%--------------------------------------------------------------------------------------------------------------------------------

}


\begin{document}

\begin{frame}
	\titlepage
\end{frame}

%--------------------------------------------------------------------------------------------------------------------------------

%%%%%%%%%%%%%%%%%%%
%End of title page%
%%%%%%%%%%%%%%%%%%%

\begin{frame}
\tableofcontents	%Leave this here (table will auto-update as you write new sections/subsubsections/appendix)
\end{frame}

%--------------------------------------------------------------------------------------------------------------------------------


\section{\textbf{Introduction}}
\subsection{\textbf{What is Quantum Computing}}

\begin{frame}{Classical Computer}
\begin{figure}[b]
	\centering
	\includegraphics[width=0.75\textwidth]{img/classical}
\end{figure}
	\begin{itemize}
		\item $x_i$ and $y_i$ are states of a 2-level classical system \{0, 1\}, bits
		\item Two equivalent models: Turing machine and circuit model
	\end{itemize}
\end{frame}

%--------------------------------------------------------------------------------------------------------------------------------


\begin{frame}{Quantum Computer}
\begin{figure}[b]
	\centering
	\includegraphics[width=0.75\textwidth]{img/quantum}
\end{figure}
	\begin{itemize}
		\item $\ket 0_i$ are states of a 2-level quantum system $\{\ket 0, \ket 1\}$, qubits
		\item One majorly used model: Circuit model
	\end{itemize}
\end{frame}

%--------------------------------------------------------------------------------------------------------------------------------


\begin{frame}{Quantum Computing}
	\begin{itemize}
		\item States are vectors in a 2-dimensional Hilbert space over $\mathbb{C}$
		\item Bases of the Hilbert space are orthonormal
			\SubItem {Most usual is 
				$\left\{
					\ket 0 = \begin{bmatrix} 1 \\ 0 \end{bmatrix},
					\ket 1 = \begin{bmatrix} 0 \\ 1 \end{bmatrix}
				\right\}$}
		\item We write the state of a qubit $\ket \phi$ as $\ket \phi = \alpha_0 \ket 0 + \alpha_1 \ket 1$
			\SubItem {$\alpha_i \in \mathbb{C}$}
			\SubItem {$\norm {\alpha_0}^2 + \norm{\alpha_1}^2 = 1 $}
			\SubItem {Measuring $\ket \phi$ yields $\ket 0$ or $\ket 1$, with probability $\norm{\alpha_i}^2$}
			\SubItem {Unitary operators are necessary to maintain these properties}
	\end{itemize}
\end{frame}

%--------------------------------------------------------------------------------------------------------------------------------


%Commented citations ccite
%\citep{feynman} \citep{quantumsimulation} \citep{shor} \citep{harrow} \citep{ibmblog} \citep{moorequantum} \citep{fukutome} \citep{} 


\subsection{\textbf{Quantum Advantages}}
\begin{frame}{Quantum Advantages}
	\begin{itemize}
		\item Quantum systems cannot be efficiently simulated on classical computers \citep{feynman}, requiring exponential complexity
		\SubItem{There are quantum algorithms \citep{quantumsimulation} that attain exponential speedups}
		\item Other problems for which quantum computers attain exponential speedups are:
			\SubItem{Factorization \citep{shor}}
			\SubItem{Solving systems of linear equations \citep{harrow}}
	\end{itemize}
\end{frame}

%--------------------------------------------------------------------------------------------------------------------------------


\subsection{\textbf{Challenges in Quantum Computing}}

\begin{frame}{Challenges in Quantum Computing}
	\begin{itemize}
		\item While quantum algorithms hold promise, current hardware leaves a lot to be desired
		\item The number of available qubits has been increasing in a similar fashion to Moore's law \citep{moorequantum}
			\SubItem{However, error rates are in the order of $0.1\%$ per elementary operation \citep{ibmblog}}
			\SubItem{Error correction is possible, but 1000 qubits are necessary to encode a single qubit with an accuracy of one in a billion \citep{moorequantum}}
	\end{itemize}
\end{frame}

%--------------------------------------------------------------------------------------------------------------------------------


\begin{frame}{Challenges in Quantum Computing}
	\begin{itemize}
		\item {When we attain $10^{-4}$ error rates, 6 million qubits would be needed to attain quantum supremacy on Shor's algorithm \citep{shor}}
		\item{For contrast, a recent study \citep{meza} has found a single bit error every month for about $2\%$ of data centers monitored}
	\end{itemize}
\end{frame}

%--------------------------------------------------------------------------------------------------------------------------------


\subsection{\textbf{Why Use Quantum Computing}}

\begin{frame}{Why Use Quantum Computing}
	\begin{itemize}
		\item There's much to be gained, but it relies on advances in quantum hardware
		\item To find uses in the near future and attract investment, NISQ (Noisy Intermediate Scale Quantum) technology is being developed
			\SubItem{Hybrid approaches - quantum computers in charge of a subset of the problem}
			\SubItem{Use shallow circuits, error tolerant quantum algorithms while relying on quantum speedups}
	\end{itemize}
\end{frame}

%--------------------------------------------------------------------------------------------------------------------------------


\subsection{\textbf{Overview}}

\begin{frame}{Overview Flowchart}
\begin{figure}[b]
	\centering
	\includegraphics[width=\textwidth]{../flowcharts/flowchart2}
\end{figure}
\end{frame}

%--------------------------------------------------------------------------------------------------------------------------------


\begin{frame}{TDVP Flowchart}
\begin{figure}[b]
	\centering
	\includegraphics[width=\textwidth]{../img/variation_figure}
\end{figure}
\end{frame}

%--------------------------------------------------------------------------------------------------------------------------------


\begin{frame}
\frametitle{Application}

%Isosurface - points with same value of spin density

%Spin Density - the probability to find an electron on a infinitesimal volume $dV$

%H2+ - electron transfer

%\includemovie[autoplay, externalviewer, text={\includegraphics{posterH2.jpg}}]{}{}{H2.mp4}
%\includemovie[poster, autoplay, externalviewer, text={\phantom{\includegraphics[width = \textwidth]{posterH2.jpg}}}]{}{}{H2.mp4}
\begin{columns}[T] % align columns
\begin{column}{.48\textwidth}
\color{black}\rule{\linewidth}{4pt}

$\text{H}_2$ molecule
\includemovie[poster={posterH2.jpg}, autoplay, externalviewer, text={\small(H2.mp4)}]{\textwidth}{6cm}{H2.mp4}
\end{column}%
\hfill%
\begin{column}{.48\textwidth}
\color{blue}\rule{\linewidth}{4pt}

$\text{H}_2^+$ molecule
\includemovie[poster={posterH2+.jpg}, autoplay, externalviewer, text={\small(H2+.mp4)}]{\textwidth}{6cm}{H2+.mp4}
\end{column}%
\end{columns}

Credit for these animations goes to Juan Dominguez from the Chemistry Department at TTU
\end{frame}

%--------------------------------------------------------------------------------------------------------------------------------


\section{\textbf{Background}}

\subsection{\textbf{Quantum Chemistry: Second quantization}}

\begin{frame}{The system}
	\begin{itemize}
		\item $\elec$ electrons
		\item $\orb$ orthonormal spin orbitals ($\orb \geq \elec$)
		\item $\{\oo{0}, \oo{1}, \oo{2}, \ldots\}$ index occupied(hole) orbitals
		\item $\{\uo{0},\uo{1},\uo{2}, \ldots\}$ index unoccupied(particle) orbitals
		\item $\{\go{0}, \go{1}, \go{2}, \go{3}, \ldots\}$ index orbitals in general.
	\end{itemize}
\end{frame}

%--------------------------------------------------------------------------------------------------------------------------------


\begin{frame}{Other considerations}
	\begin{itemize}
		\item Nuclei at rest when calculating the electron dynamics
		\item Wavefunction for each orbital: $\psig 0(\mathbf{x}_1)$, where $\bx_1 = \paren{\br_1, \sigma_1}$
			\SubItem {$\br_1$ spatial coordinate}
			\SubItem {$\sigma_1$ spin coordinate}
		\item Creation $\creg{0}$ and annihilation $\anig{0}$ operator
		\item vacuum state $\vac$
	\end{itemize}
\end{frame}

%--------------------------------------------------------------------------------------------------------------------------------


\begin{frame}{Creation and annihilation operators}
	\begin{itemize}
		\item $\creg 0 \vac = \ket{\psig 0}$
		\item $\creg 0 \creg 1 \vac = \creg 0 \ket{\psig 1} = \ket{\psig 0\psig 1}$
		\item $\creg 1 \creg 0 \vac = \creg 1 \ket{\psig 0} = \ket{\psig 1\psig 0} = - \ket{\psig 0\psig 1} = -\creg 0 \creg 1 \vac$
		\item $\anig 0 \vac = 0$
		\item $\anig 0 \ket{\psig 0}  = \vac $
		\item Anti-commutation relationships:
			\SubItem{$\anticom{\creg 0}{\anig 1} = \delta_{\go 0 \go 1}$}
			\SubItem{$\anticom{\creg 0}{\creg 1} = \anticom{\anig 0}{\anig 1} = 0$}
	\end{itemize}
\end{frame}

%--------------------------------------------------------------------------------------------------------------------------------


\begin{frame}
	\begin{itemize}
		\item Wavefunction with only occupied orbitals:
			\SubItem{$
				\ket\Psi 
				= \creo 0 \creo 1 \ldots a^\dagger_{\elec} \vac 
				= \ket {\psio 0 \psio 1 \ldots \psi_\elec} 
				= \det \left[\psio 0(\bx_i)\right]
			$}
		\item Single excited states:
			\SubItem{$
				\ket {\Psi_{\oo 0}^{\uo 0}}
				= \creu 0 \anio 0 \ket\Psi 
				= \ket{\psiu 0 \psio 1 \ldots \psi_\elec} 
			$}
		\item Double excited states:
			\SubItem{$
				\ket {\Psi_{\oo 0 \oo 1}^{\uo 0 \uo 1}}
				= \creu 1 \anio 1 \creu 0 \anio 0 \ket\Psi 
				= \ket{\psiu 0 \psiu 1 \psio 2 \ldots \psi_\elec} 
			$}
		\item Hamiltonian:
			\SubItem{$
				\hat H_e = \sum_{\go 0, \go 1}h_{\go 0 \go 1}\creg 0 \anig 1 + 
				\frac 1 4 \sum_{\go 0, \go 1, \go 2, \go 3} \sandwich{\go 0 \go 1}{}{\go 2 \go 3}\creg 0 \creg 1 \anig 2 \anig 3
			$}
			\SubItem{$
				h_{\go 0 \go 1} 
				= \sandwich {\go 0} {h} {\go 1} 
				= \int\psig 0^*(\bx_1)h(\br_1)\psig 1(\bx_1)d\bx_1 
			$}
			\SubItem{$
				\sandwich{\go 0 \go 1}{}{\go 2 \go 3}
				= \braket{\go 0 \go 1}{\go 2 \go 3} - \braket{\go 0 \go 1}{\go 3 \go 2}
			$}
			\SubItem{$
				\braket{\go 0 \go 1}{\go 2 \go 3}
				= \iint
				\psig 0^*(\bx_1)\psig 2(\bx_1) 
				\frac 1 {r_{12}}
				\psig 1^*(\bx_2)\psig 3(\bx_2) 
				d\bx_1 d\bx_2 
			$}
			\SubItem{$h(\br_1)$ is the $1e^-$ Hamiltonian and $\frac 1 {r_{12}}$ is the Coulomb repulsion operator}
	\end{itemize}
\end{frame}

%--------------------------------------------------------------------------------------------------------------------------------


\subsection{\textbf{Lie Algebras}}

\begin{frame}{Lie Algebras}
	\begin{itemize}
		\item $\orb^2$ elements in $Alg(\{\creg 1 \anig 0\}) \subset U(\orb)$
			\SubItem{Group $U(\orb)$ generated by all pair operators $\{\creg 0 \anig 1 \}$}
		%
		\item $\elec^2$ elements in $Alg(\{\creo 1 \anio 0\}) \subset U(\elec) \subset U(\orb)$
			\SubItem{Mixes holes with holes}
		\item $(\orb - \elec)^2$ elements in $Alg(\{\creu 1 \aniu 0\}) \subset U(\orb - \elec) \subset U(\orb)$
			\SubItem{Mixes particles with particles}
			\SubItem{$\left[\creo 0 \anio 1, \creu 0 \aniu 1\right] = 0$}
			\SubItem{$\implies {U(\elec) \cap U(\orb - \elec) = 0}$}
	\end{itemize}
\end{frame}

%--------------------------------------------------------------------------------------------------------------------------------


\subsection {\textbf{Fukutome Unitary Approach to HF Theory}}

\begin{frame}{Fukutome}
	\begin{itemize}
		\item Fukutome proposed a unitary approach to Hartree-Fock (HF) theory \citep{fukutome}
			\SubItem{Reduces the number of relevant variable to describe the wavefunction}
			\SubItem{Obtains a unitary representation of the wavefunction}
		\item $U(\orb) = U(\elec) \oplus_d U(\orb - \elec)$
			\SubItem{
			\(
				\hat U(\bm u) = 
				\hat U(\bm u_\xi) \cdot 
				\hat U(\bm u_\lambda) \implies 
				\bm u = 
				\bm u_\xi \cdot 
				\bm u_\lambda
				\)
			}
		\item Their actions on the reference Slater determinant state are:
			\SubItem{
				\(
					\hat U(\bm u_\xi) \ket\Psi
					= \det(\bm w)\ket\Psi \sim \ket \Psi
				\)
			}
			\SubItem{
				\(
					\hat U(\bm u_\lambda) \ket\Psi
					= \ket\Phi
				\)
			}
	\end{itemize}
\end{frame}

%--------------------------------------------------------------------------------------------------------------------------------


\section{\textbf{Methodology}}\label{chap:methodology}

\subsection{\textbf{Obtaining wavefunction}}

\begin{frame}{Considerations}
	\begin{itemize}
		\item $\elec = 2$, $\orb = 4$
		\item Use HF to obtain wavefunction $\ket\Psi = \bm u_\lambda \ket 0$.
		\item Since opposite spin orbitals don't interact:
			\( u_\lambda
			= \bigotimes_{\uo 0 \oo 0} u_{\lambda_{\uo 0 \oo 0}}
			\)
		\SubItem{\(
			\bm u_{\lambda_{\uo 0 \oo 0}} 
			= \begin{bmatrix}
				\bm C(\lambda_{\uo 0 \oo 0}) & 
				\bm -\bm S^\dagger(\lambda_{\uo 0 \oo 0}) \\ 
				\bm S(\lambda_{\uo 0 \oo 0}) & 
				\tilde {\bm C}(\lambda_{\uo 0 \oo 0})
			\end{bmatrix} \in \mathbb{\bm C}^{2\times 2}
			\)}
			\SubItem{$\bm u_{\lambda_\ind 0}$ acts on a single qubit}
		\SubItem{\(
				S(\lambda_\ind 0) 
				= \paren{\frac{\lambda_\ind 0}{\lambda_\ind 0^*}}^\frac 1 2\s{\abs{\lambda_\ind 0}} 
			\)}
		\SubItem{\(
				S^\dagger({\lambda_\ind 0}) 
				= \paren{\frac{\lambda_\ind 0^*}{\lambda_\ind 0}}^\frac 1 2\s{\abs{\lambda_\ind 0}} 
			\)}
		\SubItem{\(
				C(\lambda_\ind 0) 
				= \co {\abs{\lambda_\ind 0}} 
			\)}
		\SubItem{\(
				\tilde C({\lambda_\ind 0}) 
				= \co {\abs{\lambda_\ind 0}} 
			\)}
	\end{itemize}
\end{frame}

%--------------------------------------------------------------------------------------------------------------------------------


\begin{frame}{Decomposition}
	\begin{itemize}
		\item	\(\bm u_{\lambda_\ind 0} 
			= \begin{bmatrix}
				\co{\rho_\ind 0} & -e^{-i\omega_\ind 0}\s{\rho_\ind 0} \\
				e^{i\omega_\ind 0}\s{\rho_\ind 0} &  \co{\rho_\ind 0}
			\end{bmatrix} \)
		\SubItem {$\rho_\ind 0 = \abs{\lambda_\ind 0}$}
		\SubItem {$\omega_\ind 0 = \arg \lambda_\ind 0$}
	\end{itemize}

			\begin{equation*}
				\begin{split}
					&\bm u_{\lambda_\ind 0}
					= R_z(\omega_\ind 0)R_y(2\rho_\ind 0)R_z(-\omega_\ind 0) \\
					&= \begin{bmatrix}
						e^{-i\frac {\omega_\ind 0} 2} & 0 \\
						0 &  e^{i\frac {\omega_\ind 0} 2}
					\end{bmatrix} \cdot
					\begin{bmatrix}
						\co{\rho_\ind 0} & -\s{\rho_\ind 0} \\
						\s{\rho_\ind 0} &  \co{\rho_\ind 0}
					\end{bmatrix} \cdot
					\begin{bmatrix}
						e^{i\frac {\omega_\ind 0} 2} & 0 \\
						0 &  e^{-i\frac {\omega_\ind 0} 2}
					\end{bmatrix} \\
					& = 
					\co {\rho_\ind 0} I
					- i \s {\rho_\ind 0} \co{\omega_\ind 0} Y 
					+ i \s {\rho_\ind 0} \s{\omega_\ind 0} X
				\end{split}
			\end{equation*}
\end{frame}

%--------------------------------------------------------------------------------------------------------------------------------


\begin{frame}{The Wavefunction}
\begin{equation*}
	\begin{split}
	\ket \Psi 
	&= \bm u_{\lambda_\ind 0} \bm u_{\lambda_\ind 1} \ket {\bar 0} \\
	&= \paren{\co {\rho_\ind 0} \ket {\psi_{\oo 0}} + e^{i\omega_\ind 0}\s {\rho_\ind 0} \ket {\psi_{\uo 0}}} \\
	& \quad \otimes \paren {\co {\rho_\ind 1} \ket {\psi_{\oo 1}} + e^{i\omega_\ind 1}\s {\rho_\ind 1} \ket {\psi_{\uo 1}}} \\
	&= \co {\rho_\ind 0 }\co {\rho_\ind 1}\ket {\psi_{\oo 0}\psi_{\oo 1}} \\
	&\quad + e^{i\omega_\ind 1}\co {\rho_\ind 0 }\s {\rho_\ind 1}\ket {\psi_{\oo 0}\psi_{\uo 1}} \\
	&\quad + e^{i\omega_\ind 0 }\s {\rho_\ind 0 }\co {\rho_\ind 1}\ket{\psi_{\uo 0} \psi_{\oo 1}} \\
	&\quad + e^{i\omega_\ind 0 }e^{i\omega_\ind 1}\s {\rho_\ind 0 }\s {\rho_\ind 1}\ket{\psi_{\uo 0} \psi_{\uo 1}} 
	\end{split}
\end{equation*}
\end{frame}

%--------------------------------------------------------------------------------------------------------------------------------


%-------------------------------------------------------------------------------------

\subsection {\textbf{Dynamics}}

\begin{frame}{Dynamics: Lagrangian Equations}
	\begin{itemize}
		\item Lagrangian:
		\(
	L = \frac{\sandwich{\Psi(t)}{\frac{i}{2}\left(\overrightarrow{\pd t} - \overleftarrow{\pd t}\right) - \hat H}{\Psi(t)}}{\braket{\Psi(t)}{\Psi(t)}}
			\)
			\SubItem{$\Psi$ is normalized: $\braket \Psi \Psi = 1$}
			\SubItem{\(
				{\kpp t} = \kpp {\rho_\ind 0 }\dot \rho_\ind 0
	+	\kpp {\omega_\ind 0}\dot \omega_\ind 0
	+	\kpp {\rho_\ind 1 }\dot \rho_\ind 1
	+	\kpp {\omega_\ind 1}\dot \omega_\ind 1
			\)}
			\SubItem{We need to take $\kpp{x}$, $x \in \{\omega_\ind 1, \omega_\ind 0, \rho_\ind 1, \rho_\ind 0\}$}
		\item Euler-Lagrange equation:
		\(
				\pd{x} L = \ddt \pd{\dot x} L
			\)
		\item Combining these two, we get a system of equations:
		\SubItem{\(
			\bm M \dot{\bm \lambda} = \bm V
		\)}
	\end{itemize}
\end{frame}

%--------------------------------------------------------------------------------------------------------------------------------


\begin{frame}{Dynamics: Lagrangian Equations}
We can put the system of equations in matrix form:
\begin{equation*}
	\begin{split}
	&\scalebox{0.6}{\mbox{\ensuremath{\displaystyle 
	\begin{bmatrix}
		&0
		& \bpp{\omega_\ind 1} \kpp{\omega_\ind 0}
		- \bpp{\omega_\ind 0} \kpp{\omega_\ind 1}
		& \bpp{\omega_\ind 1} \kpp{\rho_\ind 1}
		- \bpp{\rho_\ind 1} \kpp{\omega_\ind 1}
		& \bpp{\omega_\ind 1} \kpp{\rho_\ind 0}
		- \bpp{\rho_\ind 0} \kpp{\omega_\ind 1}
		\\
		& \bpp{\omega_\ind 0} \kpp{\omega_\ind 1}
		- \bpp{\omega_\ind 1} \kpp{\omega_\ind 0}
		&0
		& \bpp{\omega_\ind 0} \kpp{\rho_\ind 1}
		- \bpp{\rho_\ind 1} \kpp{\omega_\ind 0}
		& \bpp{\omega_\ind 0} \kpp{\rho_\ind 0}
		- \bpp{\rho_\ind 0} \kpp{\omega_\ind 0}
		\\
		& \bpp{\rho_\ind 1} \kpp{\omega_\ind 1}
		- \bpp{\omega_\ind 1} \kpp{\rho_\ind 1}
		& \bpp{\rho_\ind 1} \kpp{\omega_\ind 0}
		- \bpp{\omega_\ind 0} \kpp{\rho_\ind 1}
		&0
		& \bpp{\rho_\ind 1} \kpp{\rho_\ind 0}
		- \bpp{\rho_\ind 0} \kpp{\rho_\ind 1}
		\\
		& \bpp{\rho_\ind 0} \kpp{\omega_\ind 1}
		- \bpp{\omega_\ind 1} \kpp{\rho_\ind 0}
		& \bpp{\rho_\ind 0} \kpp{\omega_\ind 0}
		- \bpp{\omega_\ind 0} \kpp{\rho_\ind 0}
		& \bpp{\rho_\ind 0} \kpp{\rho_\ind 1}
		- \bpp{\rho_\ind 1} \kpp{\rho_\ind 0}
		&0
		\\
	\end{bmatrix} }}}\\
	&\scalebox{0.6}{\mbox{\ensuremath{\displaystyle 
	\times	i
	\begin{bmatrix}
		\dot \omega_\ind 1 \\
		\dot \omega_\ind 0 \\
		\dot \rho_\ind 1 \\
		\dot \rho_\ind 0 \\
	\end{bmatrix}
	=
	\begin{bmatrix}
	&\pd{\omega_\ind 1}\sandwich{\Psi}{\hat H}{\Psi} \\
	&\pd{\omega_\ind 0}\sandwich{\Psi}{\hat H}{\Psi} \\
	&\pd{\rho_\ind 1}\sandwich{\Psi}{\hat H}{\Psi} \\
	&\pd{\rho_\ind 0}\sandwich{\Psi}{\hat H}{\Psi} \\
	\end{bmatrix} }}} \\
	&\bm M \times \dot{\bm \lambda} = \bm V
	\end{split}
\end{equation*}
\end{frame}

%--------------------------------------------------------------------------------------------------------------------------------

\begin{frame}{Dynamics: Hamiltonian for Non-Interacting Case}
Our potential matrix, $\bm V$ is:
%
\begin{equation*}
	\begin{split}
		\bm V = \begin{pmatrix}
			\pd {\lambda_1} \sandwich {\Psi}{\hat H}{\Psi}\\
			\pd {\lambda_2} \sandwich {\Psi}{\hat H}{\Psi}\\
			\pd {\lambda_3} \sandwich {\Psi}{\hat H}{\Psi}\\
			\pd {\lambda_4} \sandwich {\Psi}{\hat H}{\Psi}\\
		\end{pmatrix}
	\end{split}
\end{equation*}
%
The electrons' Hamiltonian using second quantization is \citep{szabo}:
%
\begin{equation*}
\begin{split}
	\hat H &= \sum_{\go 0, \go 1}h_{\go 0 \go 1}\creg 0 \anig 1 + 
	\frac 1 4 \sum_{\go 0, \go 1, \go 2, \go 3} \sandwich{\go 0 \go 1}{}{\go 2 \go 3}\creg 0 \creg 1 \anig 2 \anig 3
	\\
	h_{\go 0 \go 1} 
	&= \sandwich {\go 0} {h} {\go 1} 
	= \int\psig 0^*(\bx_1)h(\br_1)\psig 1(\bx_1)d\bx_1 \\
	\sandwich{\go 0 \go 1}{}{\go 2 \go 3}
	&= \braket{\go 0 \go 1}{\go 2 \go 3} - \braket{\go 0 \go 1}{\go 3 \go 2}\\
	\braket{\go 0 \go 1}{\go 2 \go 3}
	&= \iint
	\psig 0^*(\bx_1)\psig 2(\bx_1) 
	\frac 1 {r_{12}}
	\psig 1^*(\bx_2)\psig 3(\bx_2) 
	d\bx_1 d\bx_2 
\end{split}
\end{equation*}
\end{frame}

%--------------------------------------------------------------------------------------------------------------------------------

\begin{frame}{Dynamics: Hamiltonian for Non-Interacting Case}
%
	Considerations:
\begin{itemize}
	\item $\text{H}_2$ molecule under a minimal STO-3G atomic basis set
		\SubItem{In this basis set, the STOs are fitted to a fixed linear combination (“contraction”) of three primitive Gaussian functions with different orbital exponents}
	\item  Equilibrium bond distance $R = 1.4 \text{ a.u.} = 7.408478 \times 10^{-11} \text{ m}.$
	\item No electron interaction
\end{itemize}
\end{frame}

%--------------------------------------------------------------------------------------------------------------------------------

\begin{frame}{Dynamics: Hamiltonian for Non-Interacting Case}
%
\begin{equation*}
	\begin{split}
	\hat H  &= \sum_{\go 0, \go 1}h_{\go 0 \go 1}\creg 0 \anig 1 \\
		&= {h_{\alpha\alpha}} \creo 0 \anio 0
		+ {h_{\mu\mu}} \creu 0 \aniu 0
		+ h_{\mu\alpha} \creu 0 \anio 0 
		+ {h_{\beta\beta}} \creo 1 \anio 1
		+ {h_{\nu\nu}} \creu 1 \aniu 1
		+ h_{\nu\beta} \creu 1 \anio 1
%
		\\
%
		&= \frac{h_{\alpha\alpha}}{2} \paren{\hat I + \hat Z} \otimes \hat I
		+ \frac{h_{\mu\mu}}{2} \paren{\hat I - \hat Z} \otimes \hat I
		+ h_{\mu\alpha} \hat X \otimes \hat I 
%
		\\
%
		& \quad
		+ \frac{h_{\beta\beta}}{2} \hat I \otimes \paren{\hat I + \hat Z} 
		+ \frac{h_{\nu\nu}}{2} \hat I \otimes \paren{\hat I - \hat Z} 
		+ h_{\nu\beta} \hat I \otimes \hat X
%
	\\
%
		&= 	\frac{1}{2} \left[ 
			\paren{h_{\alpha\alpha} + h_{\beta\beta} + h_{\mu\mu} + h_{\nu\nu}} \hat I \otimes \hat I 
		+	\paren{h_{\alpha\alpha} - h_{\mu\mu}} \hat Z \otimes \hat I 
	\right .
	\\
		& \quad	\left .
		+	\paren{h_{\beta\beta} - h_{\nu\nu}} \hat I \otimes \hat Z 
		+ 	2 \cdot h_{\mu\alpha} \hat X \otimes \hat I 
		+ 	2 \cdot h_{\nu\beta} \hat I \otimes \hat X
	\right]
%
	\\
%
		&=	h_{II} \cdot \hat I \otimes \hat I
		+	h_{ZI} \cdot \hat Z \otimes \hat I
		+	h_{IZ} \cdot \hat I \otimes \hat Z
	\\
		& \quad 
		+	h_{XI} \cdot \hat X \otimes \hat I
		+	h_{IX} \cdot \hat I \otimes \hat X
	\end{split}
\end{equation*}
\end{frame}

%--------------------------------------------------------------------------------------------------------------------------------

\begin{frame}{Dynamics: Hamiltonian for Non-Interacting Case}
%
\begin{equation*}
	\begin{split}
			h_{II}
		&=	\frac{h_{\alpha\alpha} + h_{\beta\beta} + h_{\mu\mu} + h_{\nu\nu}} 2
%
	\\
%
			h_{ZI}
		&=	\frac{h_{\alpha\alpha} - h_{\mu\mu}} 2
%
	;
%
			h_{XI}
		= 	h_{\mu\alpha}
%
	\\
%
			h_{IZ}
		&=	\frac{h_{\beta\beta} - h_{\nu\nu}} 2
%
	;
%
			h_{IX}
		=	h_{\nu\beta}
	\end{split}
\end{equation*}
Numerical values of the one-electron spin-orbital integrals of $h_{\go 0 \go 1}$ under our assumptions are given in \citep{szabo}:
%
\begin{equation*}
	\begin{split}
		h_{\alpha\alpha} &= h_{\beta\beta} = -1.2528 \text{ a.u.} \\
		h_{\mu\mu} &= h_{\nu\nu} = -0.4756 \text{ a.u.}\\
		h_{\mu\alpha} &= h_{\nu\beta} = 0 \text{ a.u.}
	\end{split}
\end{equation*}
\end{frame}

%--------------------------------------------------------------------------------------------------------------------------------

\begin{frame}{Dynamics: Hamiltonian for Non-Interacting Case}
%
	Remember:
%
\begin{equation*}
	\begin{split}
	\hat H 
		 &=	h_{II} \cdot \hat I \otimes \hat I
		+	h_{ZI} \cdot \hat Z \otimes \hat I
		+	h_{IZ} \cdot \hat I \otimes \hat Z
	\\
		& \quad 
		+	h_{XI} \cdot \hat X \otimes \hat I
		+	h_{IX} \cdot \hat I \otimes \hat X
%
%
	\\
%
%
	\ket \Psi 
	&= 
	\co {\lambda_3}\co {\lambda_4}\ket {\psi_{\oo 0}\psi_{\oo 1}}
	+ e^{i\lambda_1}\s {\lambda_3}\co {\lambda_4}\ket{\psi_{\oo 0} \psi_{\uo 1}} 
	\\
	&\phantom{=}
	+ e^{i\lambda_2}\co {\lambda_3 }\s {\lambda_4}\ket {\psi_{\uo 0}\psi_{\oo 1}} 
	+ e^{i\lambda_1}e^{i\lambda_2}\s {\lambda_3}\s {\lambda_4}\ket{\psi_{\uo 0} \psi_{\uo 1}} 
%
	\\
%
	\bra \Psi 
	&= 
	\co {\lambda_3}\co {\lambda_4}\bra {\psi_{\oo 0}\psi_{\oo 1}}
	+ e^{-i\lambda_1}\s {\lambda_3}\co {\lambda_4}\bra{\psi_{\oo 0} \psi_{\uo 1}} 
	\\
		+& e^{-i\lambda_2}\co {\lambda_3 }\s {\lambda_4}\bra {\psi_{\uo 0}\psi_{\oo 1}} 
	+ e^{-i(\lambda_1 +\lambda_2)}\s {\lambda_3}\s {\lambda_4}\bra{\psi_{\uo 0} \psi_{\uo 1}} 
	\end{split}
\end{equation*}
\end{frame}

%--------------------------------------------------------------------------------------------------------------------------------

\begin{frame}{Dynamics: Hamiltonian for Non-Interacting Case}
Therefore:
	\begin{align*}
		\sandwich \Psi {\hat H} \Psi
		&= \sandwich {\Psi} {
			\sum_{i, j \in Pauli}
			h_{ij} i \otimes j
		} {\Psi}
%
	\\ &= \ldots \\
%
		&=	h_{II}
		+	h_{ZI} \cdot \co {2 \lambda_4}
		+	h_{IZ} \cdot \co {2 \lambda_3}
%
	\\
	&\quad
%
		+	h_{XI} \cdot \s  {2 \lambda_4} \co {\lambda_2}
		+	h_{IX} \cdot \s  {2 \lambda_3} \co {\lambda_1}
	\end{align*} %Align needed as split doesn't allow page breaks
%
Finally, we can take derivatives with respect to each $\lambda_k$ and obtain $\bm V$:
%
\begin{equation*}
	\begin{split}
		\bm V = \begin{pmatrix}
		-	h_{IX} \cdot \s  {2 \lambda_3} \s  {\lambda_1}
%
	\\
		-	h_{XI} \cdot \s  {2 \lambda_4} \s  {\lambda_2}
%
	\\
%
%
		-	2 h_{IZ} \cdot \s  {2 \lambda_3}
		+	2 h_{IX} \cdot \co {2 \lambda_3} \co {\lambda_1}
%
	\\
%
		-	2 h_{ZI} \cdot \s  {2 \lambda_4}
		+	2 h_{XI} \cdot \co {2 \lambda_4} \co {\lambda_2}
		\end{pmatrix}
	\end{split}
\end{equation*}
\end{frame}


\subsection {\textbf{Circuit}}

\begin{frame}{Goal: The Circuit}
\begin{figure}[ht!]
\includegraphics[width=\linewidth]{../circuits/circuit2}
\caption{Circuit proposed by Li \citep{benjamin} to obtain matrices $\bm M$ and $\bm V$}
\end{figure}
	\begin{itemize}
		\item{$\ket \Psi = R \ket 0 = 
			R_{N_v}(\lambda_{N_v}) 
			\ldots
			R_2(\lambda_2) 
			R_1(\lambda_1) \ket 0
		$ }
	\item{\(
				\fpd{R_k}{\lambda_k} = \sum_{i \in P} f_{k, i} R_k \sigma_{k, i}
			\)}
			\SubItem{
			$P = \{x \vert x = \bigotimes_{k = 1}^{N} x_k,\ x_k \in Pauli\}, \sigma_{k, i} \in P$}
		\item $\theta = \arg \paren{if_{k, i}^*f_{q, j}}$
		\item The average result of measuring the ancilla qubit of circuits like this lets us calculate $\bm M$ and $\bm V$
	\end{itemize}
\end{frame}

%--------------------------------------------------------------------------------------------------------------------------------


\begin{frame}{Wavefunction Decomposition}
\begin{overlayarea}{\textwidth}{\textheight}
	\begin{itemize}
		\item Decompose $\ket \Psi$ into $N_v\in\mathbb{N}$ single variable factors: 
			\SubItem{$\ket \Psi = R \ket 0 = 
			R_{N_v}(\lambda_{N_v}) 
			\ldots
			R_2(\lambda_2) 
			R_1(\lambda_1) \ket 0
		$ }
		\item In our case, $N_v = 6$ and:
		\only<1>
		{
			\SubItem{$R_6 = 
	\begin{bmatrix}
		e^{\frac{-i \omega_\ind 0} 2 } & 0 & 0 & 0 \\
		0 & e^{\frac{-i \omega_\ind 0} 2 } & 0 & 0 \\
		0 & 0 & e^{\frac{i \omega_\ind 0} 2 } & 0 \\
		0 & 0 & 0 & e^{\frac{i \omega_\ind 0} 2 }
	\end{bmatrix}
			$}
			\SubItem{$R_5 = 
	\begin{bmatrix}
		e^{\frac{-i \omega_\ind 1} 2 } & 0 & 0 & 0 \\
		0 & e^{\frac{i \omega_\ind 1} 2 } & 0 & 0 \\
		0 & 0 & e^{\frac{-i \omega_\ind 1} 2 } & 0 \\
		0 & 0 & 0 & e^{\frac{i \omega_\ind 1} 2 }
	\end{bmatrix} 
			$}
		}
		\only<2>
		{
			\SubItem{$R_4 = 
	\begin{bmatrix}
		\cos\left(\rho_\ind 0\right) & 0 & \sin\left(\rho_\ind 0\right) & 0 \\
		0 & \cos\left(\rho_\ind 0\right) & 0 & -\sin\left(\rho_\ind 0\right) \\
		\sin\left(\rho_\ind 0\right) & 0 & -\cos\left(\rho_\ind 0\right) & 0 \\
		0 & \sin\left(\rho_\ind 0\right) & 0 & \cos\left(\rho_\ind 0\right)
	\end{bmatrix}
			$}
			\SubItem{$R_3 = 
	\begin{bmatrix}
		\cos\left(\rho_\ind 1\right) & -\sin\left(\rho_\ind 1\right) & 0 & 0 \\
		\sin\left(\rho_\ind 1\right) & \cos\left(\rho_\ind 1\right) & 0 & 0 \\
		0 & 0 & -\cos\left(\rho_\ind 1\right) & \sin\left(\rho_\ind 1\right) \\
		0 & 0 & \sin\left(\rho_\ind 1\right) & \cos\left(\rho_\ind 1\right)
	\end{bmatrix}
			$}
		}
		\only<3>
		{
			\SubItem{$R_2 = 
	\begin{bmatrix}
		e^{\frac{i \omega_\ind 0} 2 } & 0 & 0 & 0 \\
		0 & e^{\frac{i \omega_\ind 0} 2 } & 0 & 0 \\
		0 & 0 & e^{\frac{-i \omega_\ind 0} 2 } & 0 \\
		0 & 0 & 0 & e^{\frac{-i \omega_\ind 0} 2 }
	\end{bmatrix}
			$}
			\SubItem{$R_1 = 
	\begin{bmatrix}
		e^{\frac{i \omega_\ind 1} 2 } & 0 & 0 & 0 \\
		0 & e^{\frac{-i \omega_\ind 1} 2 } & 0 & 0 \\
		0 & 0 & e^{\frac{i \omega_\ind 1} 2 } & 0 \\
		0 & 0 & 0 & e^{\frac{-i \omega_\ind 1} 2 }
	\end{bmatrix} 
			$}
		}
	\end{itemize}
\end{overlayarea}
\end{frame}

%--------------------------------------------------------------------------------------------------------------------------------


\begin{frame}{Obtain $f_{k, i}$:}
	\begin{itemize}
		\item Decompose the derivatives $\fpd{R_k}{\lambda_k}$ into Pauli matrices:
			\SubItem{\(
				\fpd{R_k}{\lambda_k} = \sum_{i \in P} f_{k, i} R_k \sigma_{k, i}
			\)}
			\SubItem{
			$P = \{x \vert x = \bigotimes_{k = 1}^{N} x_k,\ x_k \in Pauli\}, \sigma_{k, i} \in P$}
		\item Example for $k = 1$:
\begin{equation*}
	\begin{split}
	\frac {\partial R_1}{\partial \lambda_1}
	&= \frac {\partial R_1}{\partial \omega_\ind 1}
	=
	\begin{bmatrix}
		\frac i 2 e^{\frac{i \omega_\ind 1} 2 } & 0 & 0 & 0 \\
		0 & -\frac i 2 e^{\frac{-i \omega_\ind 1} 2 } & 0 & 0 \\
		0 & 0 & \frac i 2 e^{\frac{i \omega_\ind 1} 2 } & 0 \\
		0 & 0 & 0 & -\frac i 2 e^{\frac{-i \omega_\ind 1} 2 }
	\end{bmatrix} \\
	&= \sum_{A \in P}
		f_{1, A} \cdot R_1 \cdot A
	\end{split}
\end{equation*}
	\end{itemize}
\end{frame}

%--------------------------------------------------------------------------------------------------------------------------------


\begin{frame}{Obtain $f_{k, i}$:}
	\begin{itemize}
	\item The relevant matrices in $P$ are $\Paulid{ii}, \Paulid{iz}, \Paulid{zi}$ and $\Paulid{zz}$:
		\SubItem{\(
			\Paulid{ii} = \II,
			\Paulid{iz} = \IZ
		\)}
		\SubItem{\(
			\Paulid{zi} = \ZI,
			\Paulid{zz} = \ZZ
		\)}
\begin{equation*}
	\begin{split}
	\frac {\partial R_1}{\partial \omega_\ind 1}
	&= \sum_{A \in P}
		f_{1, A} \cdot R_1 \cdot A \\
	&= f_{1, \Paulid{ii}} R_1 \Paulid{ii}
	+ f_{1, \Paulid{iz}} R_1 \Paulid{iz} \\
	& + f_{1, \Paulid{zi}} R_1 \Paulid{zi}
	+ f_{1, \Paulid{zz}} R_1 \Paulid{zz}
	\end{split}
\end{equation*}
	\end{itemize}
\end{frame}

%--------------------------------------------------------------------------------------------------------------------------------


\begin{frame}{Obtain $f_{k, i}$:}
	\begin{itemize}
	\item {The solution to this system is:}
\begin{equation*}
	\begin{split}
	\frac {\partial R_1}{\partial \omega_\ind 1}
	= \frac i 2 R_1 \paren{\Paulid{iz}} 
	\end{split}
\end{equation*}
	\item solving for every $k \in \{1\ldots 6\}$, the non-zero $f_{k, i}$ factors are:
		\SubItem{\(
	f_{1, \Paulid{iz}} = \frac i 2 
		\)}
		\SubItem{\(
	f_{2, \Paulid{zi}} = \frac i 2
		\)}
		\SubItem{\(
	f_{3, \Paulid{iy}} = -i
		\)}
		\SubItem{\(
	f_{4, \Paulid{yz}} = i
		\)}
		\SubItem{\(
	f_{5, \Paulid{iz}} = -\frac i 2
		\)}
		\SubItem{\(
	f_{6, \Paulid{zi}} = -\frac i 2
		\)}
	\end{itemize}
\end{frame}

%--------------------------------------------------------------------------------------------------------------------------------


\begin{frame}{Expressions for $\bm M$ and $\bm V$:}
Now each individual term of $\bm M$ and $\bm V$ can be expressed as:
\begin{equation*}
\begin{split}
	M_{k, q}
	&= \sum_{m:\lambda_m=\lambda_k, n:\lambda_n=\lambda_q} \sum_{\bm l, \bm j \in P}
	\left[
	if_{m, \bm l}^*f_{n, \bm j}\sandwich 0 {\bm R_{m, \bm l}^\dagger \bm R_{n, \bm j}} 0
	+ H.C.
	\right]\\
	V_{k}
	&= \sum_{m:\lambda_m=\lambda_k} \sum_{\bm l, \bm j \in P}
	\left[
		if_{m, \bm l}^*h_{\bm j}\sandwich 0 {\bm R_{m, \bm l}^\dagger \bm \sigma_{\bm j} \bm R} 0
	+H.C.
	\right]
	\intertext{Where}
	\bm R_{k,\bm A} &= \bm R_{N_v} \ldots \bm R_k \cdot \bm A \cdot \bm R_{k - 1} \ldots \bm R_1
\end{split}
\end{equation*}
\end{frame}

%--------------------------------------------------------------------------------------------------------------------------------


\begin{frame}{Expressions for $\bm M$ and $\bm V$:}
	\begin{itemize}
		\item We know this since: 
			\SubItem{$
				M_{kq} = i \left [ \bpp{\lambda_k} \kpp{\lambda_q}
					- \bpp{\lambda_q} \kpp{\lambda_k} \right ]_{kq}
			$}
			\SubItem{$
				\kpp {\lambda_q}
				=\pd {\lambda_q} \bm R \ket 0
				=\pd {\lambda_q} \bm R_6\bm R_5\bm R_4\bm R_3\bm R_2\bm R_1 \ket 0
			$}
			\SubItem{$
				\frac {\partial R_q}{\partial \lambda_q}
				= \sum_{\bm j \in P}
				f_{q, \bm j} \cdot R_q \cdot \bm j 
			$}
			\SubItem{$
				\kpp {\lambda_q}
				=\sum_{n: \lambda_q = \lambda_n} \sum_{\bm j \in P} f_{n, \bm j}\bm R_{n, \bm j} \ket 0
			$}
			\SubItem{$
				\kpp {\lambda_k}
				=\sum_{m: \lambda_k = \lambda_m} \sum_{\bm l \in P} \bra 0 f_{m, \bm l}^* \bm R_{m, \bm l}^\dagger
			$}
		\item {Finally:}
			$$
				\bpp {\lambda_q}\kpp {\lambda_k}
				=\sum_{m: \lambda_k = \lambda_m, n: \lambda_q = \lambda_n} 
				\sum_{\bm j, \bm l \in P} 
				f_{m, \bm l}^*f_{n, \bm j}
				\sandwich{0}{\bm R_{m, \bm l}^\dagger \bm R_{n, \bm j}}{0}
			$$
	\end{itemize}
\end{frame}

%--------------------------------------------------------------------------------------------------------------------------------


\begin{frame}{Expressions for $\bm M$ and $\bm V$:}
	\begin{itemize}
		\item Furthermore:
			\[
				\paren{if_{k, i}^*f_{q, j}\sandwich 0 {R_{k, i}^\dagger R_{q, j}} 0 + H.C.} 
				= a\operatorname{Re}\paren{e^{i\theta}\sandwich 0 U 0}
			\]
			\SubItem{When taking $a = \abs{if_{k, i}^*f_{q, j}}, \theta = \arg \paren{if_{k, i}^*f_{q, j}}$ and $U = {R_{k, i}^\dagger R_{q, j}}$}
		\item Measuring the ancilla qubit on the circuit gives us $\operatorname{Re}\paren{e^{i\theta}\sandwich 0 U 0}$
			\SubItem{We can then obtain $\bm M$ via weighed sums of the average measurements of the circuits}
	\end{itemize}
\end{frame}

%--------------------------------------------------------------------------------------------------------------------------------


\begin{frame}{Minimal Example Circuit}
\begin{overlayarea}{\textwidth}{\textheight}
\only<1-2>
{
\begin{figure}[ht!]
  \includegraphics[width=.75\linewidth]{../circuits/circuit1}
\end{figure}
}

\begin{itemize}
\only<1>
{
	\item \(
		\ket{\psi_0} 
		= \paren{\frac{\ket 0 + e^{i\theta}\ket 1}{\sqrt 2}}\otimes \ket 0^{\otimes\elec}
		\)
	\item \(
		\ket{\psi_1}
		= \paren{\frac{\ket 1 + e^{i\theta}\ket 0}{\sqrt 2}}\otimes \ket 0^{\otimes\elec} 
		\)
	\item \(
		\ket{\psi_2}
		= \frac{\ket 1}{\sqrt 2} \otimes U \ket 0^{\otimes\elec} + \frac{e^{i\theta}}{\sqrt 2}\ket 0 \otimes \ket 0^{\otimes\elec} 
		\)
  }
	\only<1-2>
	{
	\item \(
		\ket{\psi_3}
		= \frac{\ket 0}{\sqrt 2} \otimes U \ket 0^{\otimes\elec} + \frac{e^{i\theta}}{\sqrt 2}\ket 1 \otimes \ket 0^{\otimes\elec} 
		\)
	}

	\only<2>
	{
			Changing basis on the first qubit to 
			$\ket + = \frac{\ket 0 + \ket 1}{\sqrt 2}$
			and
			$\ket - = \frac{\ket 0 - \ket 1}{\sqrt 2}$:
			\begin{equation*}
				\begin{split}
					\psi_3
					&= \frac{\ket + + \ket -}{2} \otimes U \ket 0^{\otimes\elec} + e^{i\theta}\frac{\ket + - \ket -}{2} \otimes \ket 0^{\otimes\elec} \\
					&= \ket + \frac{U + e^{i\theta}I}{2}\ket 0^{\otimes\elec}
					+ \ket - \frac{U - e^{i\theta}I}{2}\ket 0^{\otimes\elec}
				\end{split}
			\end{equation*}
	}
\end{itemize}
\end{overlayarea}
\end{frame}

%--------------------------------------------------------------------------------------------------------------------------------

\begin{frame}{Minimal Example Circuit}
	\begin{itemize}
		\item $\ket{\psi_3}
					= \ket + \frac{U + e^{i\theta}I}{2}\ket 0^{\otimes\elec}
					+ \ket - \frac{U - e^{i\theta}I}{2}\ket 0^{\otimes\elec}$
			\SubItem{The probability of measuring $\ket +$ on the first qubit is:}
\begin{equation*}
	\begin{split}
		P(\ket +)
		&= \sandwich 0 {
			\paren{\frac{U + e^{i\theta}I}{2}}
			\paren{\frac{U + e^{i\theta}I}{2}}^\dagger
		} 0 \\
		&= \frac{Re \paren{e^{i\theta}\sandwich 0 { U } 0}}{2}
		+ \frac{1}{2} \\
	\end{split}
\end{equation*}
			\SubItem{The probability of measuring $\ket -$ is then:}
\begin{equation*}
	\begin{split}
		P(\ket -)
		&= \frac{1}{2} - \frac{Re \paren{e^{i\theta}\sandwich 0 { U } 0}}{2}
	\end{split}
\end{equation*}
	\end{itemize}
\end{frame}

%--------------------------------------------------------------------------------------------------------------------------------

\begin{frame}{Minimal Example Circuit}
	\begin{itemize}
		\item Since the eigenvalue of $\ket +$ is 1 and the eigenvalue of $\ket -$ is -1, the average measurement is then:
			\SubItem{$Avg = P(\ket +) \cdot 1 + P(\ket -) \cdot (-1) = Re\paren{e^{i\theta}\sandwich 0 {U} 0}$}

		\vspace{10pt}
		\item This was a simple circuit showcasing how to obtain the expression needed. The actual circuit is larger, but uses the same general idea
\begin{equation*}
	\begin{split}
	\end{split}
\end{equation*}
	\end{itemize}
\end{frame}

%--------------------------------------------------------------------------------------------------------------------------------


\begin{frame}{The Circuit}
\begin{figure}[ht!]
\includegraphics[width=\linewidth]{../circuits/circuit2}
\end{figure}
	\begin{itemize}
		\item Same as in the last circuit, we measure the first qubit:
\begin{equation*}
	\begin{split}
		\ket \phi
		&= \frac{\ket 0}{\sqrt 2} \otimes R_{N_v} \ldots R_k \cdot \sigma_{ki} \cdot R_{k-1} \ldots R_1 \ket 0^{\otimes \elec}\\
		&\quad + \frac{e^{i\theta}\ket 1}{\sqrt 2} \otimes R_{N_v} \ldots R_q \cdot \sigma_{qj} \cdot R_{q-1} \ldots R_1 \ket 0^{\otimes \elec}\\
		&= \frac{\ket 0}{\sqrt 2} \otimes R_{ki} \ket 0^{\otimes \elec}
		+ \frac{e^{i\theta}\ket 1}{\sqrt 2} R_{qj}\ket 0^{\otimes \elec}\\
		\span\text{Changing basis:}\\
		&= \frac{\ket + + \ket -}{2} \otimes R_{ki} \ket 0^{\otimes \elec}
		+ \frac{e^{i\theta}\paren{\ket + - \ket -}}{2} R_{qj}\ket 0^{\otimes \elec}
	\end{split}
\end{equation*}
	\end{itemize}
\end{frame}

%--------------------------------------------------------------------------------------------------------------------------------

\begin{frame}{The Circuit}
\begin{equation*}
	\begin{split}
		\ket\phi = \ket + \otimes \frac{R_{ki} + e^{i\theta}R_{qj}}{2} \ket 0^{\otimes \elec}
		 + \ket - \otimes \frac{R_{ki} - e^{i\theta}R_{qj}}{2} \ket 0^{\otimes \elec}
	\end{split}
\end{equation*}
\begin{itemize}
	\item The probability of measuring $\ket +$ is:
\begin{equation*}
	\begin{split}
		P(\ket +)
		&= \sandwich 0 {
			\frac{1}{2}\paren{R_{ki} + e^{i\theta} R_{qj}}
			\frac{1}{2}\paren{R_{ki} + e^{i\theta} R_{qj}}^{\dagger}
		} 0 \\
		&= \frac 1 2 Re\paren{e^{i\theta}\sandwich 0 {
			 R_{ki}R_{qj}^{\dagger}
			} 0 }
			+ \frac{1}{2}\\
	\end{split}
\end{equation*}
	\item The probability of measuring $\ket -$ is then:
\begin{equation*}
	\begin{split}
		P(\ket -)
		= \frac 1 2 - \frac 1 2 Re\paren{e^{i\theta}\sandwich 0 {
			 R_{ki}R_{qj}^{\dagger}
	} 0 }
	\end{split}
\end{equation*}
\end{itemize}
\end{frame}

%--------------------------------------------------------------------------------------------------------------------------------


\begin{frame}{The Circuit}
	\begin{itemize}
		\item The average measurement of the circuit is:
			$$Avg = P(\ket +) \cdot 1 + P(\ket -) \cdot (-1) = Re\paren{e^{i\theta}\sandwich 0 {R_{ki}R_{qj}^{\dagger}} 0}$$
		\item Taking $a_{kiqj} = \abs{if_{k, i}^*f_{q, j}}, \theta = \arg \paren{if_{k, i}^*f_{q, j}}$ and $R_{ki}R_{qj}^{\dagger}$:
\begin{equation*}
	\begin{split}
		\sum_{i, j \in P} a_{kiqj} \cdot Avg_{kiqj}
		&= \sum_{i, j \in P} a_{kiqj}\cdot Re\paren{e^{i\theta}\sandwich 0 {R_{ki}R_{qj}^{\dagger}} 0}\\
		&= \sum_{i, j \in P}
		\left [if_{k, i}^*f_{q, j}\sandwich 0 {R_{ki}R_{qj}^{\dagger}} 0 +
		H.C. \right ] \\ 
		&= M_{k, q} \\
	\end{split}
\end{equation*}
	\end{itemize}
\end{frame}

%--------------------------------------------------------------------------------------------------------------------------------


\section{\textbf{Results and Research Directions}}

\subsection{\textbf{Results}}

\begin{frame}{Error Graph - 1 electron}
\begin{figure}[b]
	\centering
	\includegraphics[width=\textwidth]{../img/1e.jpg}
  \caption{Loglog graph of error vs number of shots for a $\text{H}_2^+$ hydrogen molecule}
  \label{fig:1derrorgraph}
\end{figure}
\end{frame}

%--------------------------------------------------------------------------------------------------------------------------------


\begin{frame}{Error Graph - 2 electrons}
\begin{figure}[b]
	\centering
	\includegraphics[width=\textwidth]{../img/2e-nonint.jpg}
  \caption{Loglog graph of error vs number of shots for a $\text{H}_2$ hydrogen molecule}
  \label{fig:2derrorgraph}
\end{figure}
\end{frame}

%--------------------------------------------------------------------------------------------------------------------------------


\subsection{\textbf{Conclusion}}

\begin{frame}{Contributions}
	\begin{itemize}
		\item United HF method with other approximations to obtain a unitary representation of a trial wavefunction for the system

		\item Decomposed said wavefunction into elementary rotations

		\item Further decomposed the wavefunction into a product of matrices that depend on a single variable

		\item Solved the Euler-Lagrange differential equation system for specific cases so we can verify our results

		\item Modified Li's circuit \citep{benjamin} that calculates the elements of the matrices $\bm M$ and $\bm V$ from the differential equation system to use in NISQ equipment

		\item Verified the theoretical output of said circuit
	\end{itemize}
\end{frame}

%--------------------------------------------------------------------------------------------------------------------------------

\begin{frame}{Applications}
	\begin{itemize}
		\item The TDVP \citep{kramer} has been used to simulate many physical systems, such as:
			\SubItem{Nuclear reactions \citep{tdvp4}}
			\SubItem{Molecular models \citep{tdvp5,kramer}}
			\SubItem{Wave packet dynamics in the multi-configuration Hartree method \citep{tdvp7,tdvp8,tdvp9,tdvp10,tdvp11}, which is used to simulate non-adiabatic processes in molecules and molecular reactions.}
	\end{itemize}
\end{frame}

%--------------------------------------------------------------------------------------------------------------------------------

\begin{frame}{Applications}
\begin{overlayarea}{\textwidth}{\textheight}
	\begin{itemize}
		\item In conjunction to other techniques, such as END \citep{end1,end2,end3}, and its derivatives, SLEND and MCEND:
		\only<1>{
			\SubItem{Simulate molecules under intense electromagnetic fields \citep{end12}}
			\SubItem{Molecule collisions leading to moieties' abstractions and exchanges \citep{end13}}
			\SubItem{Reactions in organic chemistry \citep{end2}, rotations in triatomic molecules \citep{end14}}
			\SubItem{Stopping powers and energy losses in penetration processes of ions into material media \citep{end15,end16,end17,end18}}
		}
		\only<2>{
			\SubItem{Proton-molecule reactions of interest in atmospheric chemistry and astrochemistry; $ \text{H}^+$ and the following molecules: $ \text{H}_2$ \citep{end19,end20}, C$ \text{H}_4$ \citep{end21}, $ \text{H}_2$O\citep{end22,end23,end24}, $ \text{N}_2$ \citep{end25}, NO\citep{end26}, CO\citep{end27}, $ \text{C}_2$$ \text{H}_2$ \citep{end29}, $ \text{C}_2$$ \text{H}_4$ \citep{end30}}
		}
		\only<3>{
			\SubItem{Simulate ion cancer therapy reactions, like ion-induced water radiolysis \citep{end2,end24,end31,moraleswater,moralesdna}, ion-induced nucleobase damage \citep{end2,end31,moralesdna,end34} and ion-induced and electron-induced nucleotide damage \citep{moralesdna}}
			\SubItem{Simulation of time-dependent symmetry breaking during molecular collisions \citep{end24,end35}.}
		}
	\end{itemize}
\end{overlayarea}
\end{frame}

%--------------------------------------------------------------------------------------------------------------------------------


\subsection{\textbf{Research directions}}

\begin{frame}{Future Directions}
	\begin{itemize}
		\item Use of non-unitary representations of $\hat U$ in combination with Linear Combination of Unitaries (LCU) to make it possible to use quantum computation
		\item Apply to systems with special geometry
			\SubItem{Crystals}
			\SubItem{Polymers}
		\item Apply Quantum Approximate Optimization Algorithm (QAOA)
		\item Use Variational Quantum Eigensolver (VQE)
		\item Prepare code to submit to a real quantum computer
	\end{itemize}
\end{frame}

\begin{frame}{Questions}
	\begin{center}
		\Huge Questions?
	\end{center}
\end{frame}

%--------------------------------------------------------------------------------------------------------------------------------

\begin{frame}{Acknowledgements}
	\begin{itemize}
		\item Dr. Ismael Regis de Farias Jr.
		\item Dr. Bryan A. Norman
		\item Dr. Jorge Alberto Morales
		\item Dr. Jorge Alberto Morales' Research Group
			\SubItem {In particular Juan Dominguez}
		\item TTU High Performance Computing Center
		\item Edward E. Whitacre Jr. College of Engineering
	\end{itemize}
\end{frame}

%--------------------------------------------------------------------------------------------------------------------------------

\begin{frame}{FIN}
	\begin{center}
		\Huge Thank You!
	\end{center}
\end{frame}

%--------------------------------------------------------------------------------------------------------------------------------

\bibliographystyle{chicago}

\section{\textbf{References}}
\bibliography{refs}

\end{document}
